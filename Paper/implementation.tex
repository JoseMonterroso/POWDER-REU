% -*- mode: LaTeX -*-
%%

%%%%%%%%%%%%%%%%%%%%%%%%%%%%%%%%%%%%%%%%%%%%%%%%%%%%%%%%%%%%%%%%%%%%%%%%%%%%%%%

\section{Implementation}
\label{sec:implementation}

Now we will proceed to discuss the implementation behind our design.
We will discuss in detail what our work uses to achieve service specific
routing at the intra- and inter-domain topologies. 

\textbf{BGP} BGP is used to support prefix advertisement along with the usage
of BGP large community attribute. Where the large community attribute is designed
to be seen as number:number:number. Where the first number is the global identifier,
and the second and third numbers respectively represent the service and subtype of a service. 

\textbf{Free Range Routing} We choose Free Range Routing (FRR) to be our
routing stack. We will be running IPv6 in our network. The daemons we have chosen 
to use in FRR are Zebra, BGP, and OSPFv3. Zebra is an ip routing manger that
provides us with routing table updates, interface lookups, and redistribution of routes
between different routing protocols. Next we used BGP, specifically BGP version 4. 
Lastly, we used OSPFv3 that supports OSPF routing for an IPv6 network. 

\textbf{Segment Routing} For the adoption of segment routing into our network we 
we used a linux kernel and set various sysctls to enable segment routing.~\cite{SRV6}

\textbf{SDN} We created our own application to be able to read BGP prefixes, and acquire their 
respective large community attribute. From there we had associated segment routing rules 
that would be pushed into the router depending on the service type the large community 
attribute defined for the prefix.

%%%%%%%%%%%%%%%%%%%%%%%%%%%%%%%%%%%%%%%%%%%%%%%%%%%%%%%%%%%%%%%%%%%%%%%%%%%%%%%

%% End of file.