% -*- mode: LaTeX -*-
%%

%%%%%%%%%%%%%%%%%%%%%%%%%%%%%%%%%%%%%%%%%%%%%%%%%%%%%%%%%%%%%%%%%%%%%%%%%%%%%%%

\section{Introduction}
\label{sec:introduction}


Today's internet plays a vital role in providing a multitude of services for the user. With all traffic
being treated the same, the internet demands a more reliable method of routing. Today's routing
involves the 'message in a bottle' approach. In this approach we heavily rely on routing tables to 
push packets towards the right destination. No matter what type of service the traffic may be, all 
traffic is treated the same. In this primitive approach there is no sense of urgency or priority. 
Hence a more granular approach over today's routing is needed more than ever. 
 
In past attempts within an Autonomous System (AS) we have seen Multiprotocol Label Switching
(MPLS)~\cite{MPLSSDN}. In MPLS traffic is directed from one node to the next by the use of short
path labels rather than long network addresses. We have also seen Software Defined Networking
(SDN)~\cite{SDNNFV} approaches to service specific routing where an individual packet header's
attributes are matched and an action is imposed on the packet. Segment Routing (SR)~\cite{SRR}
along side IPv6 is the newest in routing paradigms. In SR using IPv6, an ordered list of segments is encoded
in a routing extension header. The header contains a list of segments that guide the packet through
the AS. 

Similarly on the Interdomain level we have seen Software Defined Internet Exchange Points (SDXs) ~\cite{SDX1, SDX2}.
SDXs bring the ideas behind SDN to interdomain routing. Located between
ASes, SDXs aid in service specific routing by allowing the network to execute a far wider range of 
decisions concerning end-to-end traffic delivery. There also exists service specific routing across 
domains through the use of SDN~\cite{InterDSDN}. The idea is to form a logically centralized 
controller running routing processes over multiple ASes. The controller would be aware of (parts of)
the policies, and topologies of the ASes. We have also seen approaches at the interdomain level 
that involve using route servers~\cite{RouteServers}. Router servers located at the heart of an
internet exchange point (IXP) are key enablers for offering new and complex peering options to
IXP members.
 
In this paper we will define a new approach to service specific routing. In our approach we use 
Border Gateway Protocol (BGP) and segment routing to create a novel service specific
interdomain and intradomain routing framework. BGP is the primary form of interdomain 
communication amongst ASes. As such it's extremely difficult to create a new interdomain 
protocol because every single AS would need to change in order to enable full internet
connectivity. Therefore, we have chosen to remain in the realm of BGP. The combination of 
BGP working at the control plan and SR working at the data plan enables a fine grained control 
of the treatment of service specific traffic. 

The contributions of this work are the following. First, we propose a new approach to service
specific routing involving BGP's large community attribute to inform ASes on how to handle the
traffic of an advertised prefix. The ASes will use SR to guide the traffic in a matter they see fit
regarding the type of traffic that is being guided. Secondly, we develop a proof of concept 
demonstration using the POWDER testbed. 

%%%%%%%%%%%%%%%%%%%%%%%%%%%%%%%%%%%%%%%%%%%%%%%%%%%%%%%%%%%%%%%%%%%%%%%%%%%%%%%

%% End of file.




